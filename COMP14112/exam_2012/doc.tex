\documentclass{article}

\usepackage{fancyhdr}
\usepackage[parfill]{parskip}
\usepackage{enumerate}
\usepackage{amsmath,amssymb}

\pagestyle{fancyplain}

\author{Todd Davies}
\title{COMP14112 exam 2012}
\date{\today}

\begin{document}

\rhead{COMP14112 exam 2012}
\lhead{\today}

\maketitle

\begin{enumerate}
  
  \item

  \begin{enumerate}[(a)]
    \item {\bf In the robot localization problem, why is it impossible to
           determine the exact pose that the robot holds? Due to this
           impossibility, how is the location of the robot
           represented?}\\
      
      The robot's sensors are noisy, and when the robot moves, there is no
      garentee that it moves by the amount that was intended. Because of
      this, there is always some degree of uncertainty in the data
      regarding the robot, and therefore no absolute conclusions can be
      drawn.

      Because of this, the robots position is represented as a probability
      matrix, where each pose has a probability assigned to it to indicate
      how likely it is that the robot is occupying that position. The sum
      of the probability of all the pose's is 1.

    \item {\bf Use the definition of probability distribution to prove that
           $\mathbb{P}(E \vee F) = \mathbb{P}(E) +
           \mathbb{P}(F \wedge E^C)$ where $E^C$ is the complement of
           $E$.}\\

      \[
        \begin{split}
         \mathbb{P}(E \vee F) &= \mathbb{P}(E) + \mathbb{P}(F) - \mathbb{P}(E \wedge F)\\
         \mathbb{P}(F) &= \mathbb{P}(F \wedge E) + \mathbb{P}(F \wedge E^C)\\\\
         \mathbb{P}(E) + \mathbb{P}(F) - \mathbb{P}(E \wedge F) &= \mathbb{P}(E) + \mathbb{P}(F \wedge E^C)\\
         \mathbb{P}(F) - \mathbb{P}(E \wedge F) &=\mathbb{P}(F \wedge E^C)\\
         \mathbb{P}(F) &=\mathbb{P}(F \wedge E^C) + \mathbb{P}(E \wedge F)\\
         \mathbb{P}(F) &=\mathbb{P}(F)\\
        \end{split}
      \]

    \item {\bf In the robot localization problem, suppose that the 1st
               observation received from a sensor is $o_1$ and the second
               observation is $o_2$ , and the probabilities of all poses are
               updated every time when an observation is received. Are the final
               probabilities obtained the same as the probabilities obtained by
               using $o_2$ updating first and $o_1$ second? Please justify your
               answer.}

    \item {\bf Let a single robot be at a point object located at some position
               in a square arena with a 2×2 square grid, in which there is one
               obstacle occupying position (1, 1). Now answer the following
               questions:}

      \begin{enumerate}[(i)]
        \item $\frac{1}{(2 \cdot 2) - 1} = \frac{1}{3}$
      \end{enumerate}

  \end{enumerate}


\end{enumerate}

\end{document}
