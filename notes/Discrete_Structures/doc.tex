\documentclass{article}

\usepackage{fancyhdr}
\usepackage[parfill]{parskip}
\usepackage{amsfonts}

\pagestyle{fancyplain}

\author{Todd Davies}
\title{Discrete Structures}
\date{\today}

\begin{document}

\rhead{Discrete Structures}
\lhead{\today}

\maketitle

\section{Introduction}

\subsection{Terminology}

A {\it structure} consists of certain {\it sets}. It also contains {\it elements} of these sets, {\it operations} on these sets and {\it relations} on these sets.

\subsection{Carriers to learn}

The following carriers must be learnt:

$\mathbb{N}$ The set of natural numbers (all whole numbers from $0$ to $\infty$)

$\mathbb{Z}$ The set of integers (all whole numbers from $-\infty$ to $\infty$)

$\mathbb{Q}$ The set of rational numbers (any integer divided by any other integer - e.g. $\frac{5}{4}=1.25$)

$\mathbb{R}$ The set of real numbers (all finite and infinite decimal numbers)

\end{document}
