\section*{Introduction}

This course covers the fundamental maths required by Computer Science students
in order to successfully complete the reminder of their courses as well as for a
career in computer science. It includes modules on discrete structures, set
theory, logic, probability, mathematical induction, relations, vectors, matrices
and transformation.

\section*{Aims}

This full-year course unit focuses on the use of mathematics as a tool to model
and analyse real-world problems arising in computer science. Four principal
topics, drawn from the traditional areas of discrete mathematics as well as some
continuous mathematics, will be introduced: symbolic logic, probability,
discrete structures, and vectors and matrices. Each topic will be motivated by
experts introducing relevant real-world problems arising in their own
specialism.

Abstraction is fundamental to computer science. Hence, a fundamental emphasis of
this course unit is to introduce mathematical techniques and skills to enable
the student to design and manipulate tractable and innovative abstract models of
chunks of the real-world. These techniques and skills include appropriate
mathematical notations and concepts. These range over the four principal areas
mentioned above. Formalisation in mathematics has, in general, significant cost.
Therefore, to be of practical use, the benefits arising from formalisation, such
as succinctness, unambiguity, provability, transformability and mechanisability,
must outweigh the costs. A key aim of the course is for the student to
appreciate this issue and know how and when to use particular techniques.

The specific aims of the course unit are:

\begin{itemize}
	\item To demonstrate the relevance of mathematics to computer science.
	\item To introduce fundamental mathematical techniques of abstraction.
	\item To demonstrate applicability of particular mathematical techniques and
		  skills for particular types of computer science problem.
	\item To appreciate the costs and benefits of mathematical modelling.
	\item The delivery style will place more emphasis on students undertaking
	      appropriate background reading, i.e. being more independent learners,
	      and use the lectures more to demonstrate examples and solutions and
	      not working through every detail of a particular or concept.
\end{itemize}

The course unit is delivered by staff from both the School of Computer Science
and the School of Mathematics.

\section*{Licence and contrubution}

These notes are based off the material from the COMP11110 course run by 
Dr. Aravind Vijayaraghavan. They are released
under a Creative Commons licence, please submit issues and pull requests at
\url{https://github.com/Todd-Davies/first-year-notes}.