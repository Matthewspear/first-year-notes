% Set the author and title of the compiled pdf
\hypersetup{
	pdftitle = {\Title},
	pdfauthor = {\Author}
}

\section*{Functions}

A mathematical function is a relation between a set of inputs and a set of
outputs. Each input is related to exactly one output, and the same input will
{\it always} result in the same output.

\subsection*{Simple functions}

Functions can be split into three components; a source, a target and the
behaviour of the function.

The source and target of a function are both sets, and are denoted as $S$ and
$T$ respectively in the following notation:

\[
	f:S \rightarrow T
\]

In order to find the value of a function $f$ for a specific input $x$, the
notation is:

\[
	f(x)
\]

Describing the behaviour of a function requires another notation:

\[
	x \longmapsto \dots x \dots
\]

Concequently, to completely describe a function by defining it's source, target
and behaviour, we must write something like:

\[
	\begin{split}
	f:S &\rightarrow T\\
	x &\longmapsto \dots x \dots
	\end{split}
\]

An example of this may be a function that finds the absolute value of integers:

\[
	\begin{split}
	abs:\mathbb{Z} &\rightarrow \mathbb{N}\\
	x &\longmapsto |x|
	\end{split}	
\]

\subsection*{Equality of functions}

Two functions are said to be equal to each other if the following conditions are
met:

\begin{itemize}
	\item The source set is the same
	\item The target set is the same
	\item The behavior of the functions is the same (so for every $x \in S$, 
		  where $S$ is the source set, both functions give the same output $y$
		  when given $x$ as the input, where $y \in T$ and $T$ is the target
		  set).
\end{itemize}

\subsection*{Identity functions}

For any set $S$, there is a function such that:

\[
	\begin{split}
	f:S \rightarrow S\\
	x \longmapsto x
	\end{split}
\]	

This is described as the identity function of $S$, and is often denoted by $1_S$
or $id_S$

\subsection*{Injectivity, Surjectivity and Bijectivity}

The terms {\it injective}, {\it surjective} and {\it bijective} are used to
describe which elements in a functions source set are mapped to which elements
it's target set:

\begin{description}
	\item {\bf Injective}\\
		For each $x \in S$ there is at most one $y \in T$ such that $f(x) = y$.
	\item {\bf Surjective}\\
		For each $x \in S$ there is at least one $y \in T$ such that $f(x) = y$.
	\item {\bf Bijective}\\
		For each $x \in S$ there is only one $y \in T$ such that $f(x) = y$.
\end{description}

Note that if an function is injective and surjective, then it must also be
bijective.

\subsubsection*{Permutation of sets}

A function is said to be a permutation of a set if the source and target of the
function is the same set and the function is bijective. Applying the function
effectively re-arranges the members of the set.

\aubsection*{Function composition}

Two functions can be composted when the source set of one function is the same
as the target set of another. For example:

\[
	\begin{split}
		f:S \rightarrow T\\
		g:T \rightarrow U
	\end{split}
\]

Here, we can compose a new function made up of $f$ and $g$:

\[
	h = g \circ f
\]

Or in other words:

\[
	h(x) = g(f(x))
\]

When we're invoking composition on more than two functions, it is an associative
operation. For this reason, we often omit brackets and the circle operator:

\[
	(f \circ (g \circ h)) = fgh
\]	
