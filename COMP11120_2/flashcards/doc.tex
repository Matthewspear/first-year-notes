\documentclass[frontgrid]{flacards}
\usepackage{color}
\usepackage{amsmath}

\definecolor{light-gray}{gray}{0.75}

\newcommand{\frontcard}[1]{\textcolor{light-gray}{\colorbox{light-gray}{$#1$}}}
\newcommand{\backcard}[1]{#1} 

\newcommand{\flashcard}[1]{% create new command for cards with blanks
  \card{% call the original \card command with twice the same argument (#1)
    \let\blank\frontcard% but let \blank behave like \frontcard the first time
    #1
  }{%
    \let\blank\backcard% and like \backcard the second time
    #1
  }%
}

\begin{document}

\pagesetup{2}{4}

%===============================================================================
% Matrices basics
%===============================================================================

\card{
  How do you add two matrices together? What are the conditions for matrix
  addition?
}{
  In order for you to be able to add two matrices together, they must both
  have the same dimensions. Then, just add value of each position in one
  matrix to the corresponding position in the other matrix.

  \[
    (A + B)_{ij} = A_{ij} + B_{ij}
  \]
}

\card{
  How can a matrix be scaled?
}{
  Just multiply each cell in the matrix by the scaling factor.
}

\card{
  How is matrix subtraction performed? What are the conditions?
}{
  In order to subtract one matrix from another, just scale the first one by a
  factor of $-1$ and add them together. As with addition, the matrices must
  have the same dimensions.

  \[
    (A - B)_{ij} = (A + (-1)B)_{ij}
  \]
}

\card{
  What is the condition for matrix multiplication?
}{
  The number of columns in the first matrix must equal the number of rows in
  the second.
}

\card{
  How do we determine the value of the cell $i,j$ when multiplying a matrix
  $A$ by another matrix $B$?
}{
  It depends on the values in the $i$'th row of $A$ and the $j$'th column of
  $B$.
  \[
    C_{ij} = \sum^{n}_{k=1} A_{ik}B_{kj}
  \]
}

\card{
  What is the identity matrix?
}{
  
  A square matrix where every cell is set to zero, except from those on the
  diagonal from the top left to the bottom right, where they are set to one.
  For example:
  \[
    \left(
      \begin{array}{ccc}
        1 & 0 & 0\\
        0 & 1 & 0\\
        0 & 0 & 1\\
      \end{array}
    \right)
  \]
}

%\card{
% How do you find the inverse of a matrix?
%}{
% In order to find the inverse of a matrix, you scale it be $-1$ and then
% multiply it by itself:
% \[
%   A^{-1} = A(-1)A
% \]
%}

\card{
  How do you work out the transpose of a matrix? What is the notation?
}{
  To work out the transpose of a matrix, you simply rotate everything
  clockwise by $90^{\circ}$. E.g.:
  \[
    \begin{split}
    A &= \left(
        \begin{array}{cc}
          1 & 3\\
          0 & 5\\
          8 & 7\\
        \end{array}
      \right)\\
    A^T &= \left(
        \begin{array}{ccc}
          1 & 0 & 8\\
          3 & 5 & 7\\
        \end{array}
      \right)
    \end{split}
  \]
}

\card{
  What is a symmetric matrix?
}{
  A matrix $M$ is symmetric when $M = M^T$. This means it is symmetric along
  the main diagonal.\\\vspace{1em}
  Alternately, you could say that $M_{ij} = M_{ji}$.
}

\card{
  Define the zero matrix.
}{
  A matrix where all of the cells have a value of zero.

  \[
    \left(
      \begin{array}{ccc}
        0 & 0 & 0\\
        0 & 0 & 0\\
        0 & 0 & 0\\
      \end{array}
    \right)
  \]
}

\card{
  Define the term {\bf commuting matrices}.
}{
  A pair of matrices $A,B$ where $AB = BA$
}

\card{
  Is matrix multiplication associative?
}{
  Yes.
  \[
    A(BC) = (AB)C
  \]
}

\card{
  Is matrix multiplication a distributive operation?
}{
  Yes.
  \[
    \begin{split}
      A(B + C) = AB + AC\\
      (B + C)A = BA + CA
    \end{split}
  \]
}

\card{
  Does the following hold between the two matrices $A$ and $B$:
  \[
    (AB^T) = A^TB^T
  \]
}{
  Yes.
}

\card{
  Does the following matrix represent a point or a vector?
  \[
    \left[
      \begin{array}{c}
        4\\
        2\\
        3\\
        1
      \end{array}
    \right]
  \]
}{
  The point $(4,2,3)$.
}

\card{
  Does the following matrix represent a point or a vector?
  \[
    \left[
      \begin{array}{c}
        4\\
        2\\
        3\\
        0
      \end{array}
    \right]
  \]
}{
  The vector $4i + 2j + 3k$.
}

%===============================================================================
% Affine transformations and matrices
%===============================================================================

\card{
  What cells are the same for all affine transformation matrices?
}{
  The bottom three/four.
  \[
    \left(
      \begin{array}{cccc}
        a_{11} & a_{12} & a_{13} & b{1}\\
        a_{21} & a_{22} & a_{23} & b{2}\\
        a_{31} & a_{32} & a_{33} & b{3}\\
        0      & 0      & 0      & 1
      \end{array}
    \right)
    ~and~
    \left(
      \begin{array}{ccc}
        a_{11} & a_{12} & b{1}\\
        a_{21} & a_{22} & b{2}\\
        0      & 0      & 1
      \end{array}
    \right)
  \]
}

\card{
  What is the matrix that will perform an affine translation?
}{
  \[
    \left(
      \begin{array}{cccc}
        1 & 0 & 0 & x\\
        0 & 1 & 0 & y\\
        0 & 0 & 1 & z\\
        0 & 0 & 0 & 1\\
      \end{array}
    \right)
  \]
}

\card{
  What is the matrix that will perform an affine scaling?
}{
  \[
    \left(
      \begin{array}{cccc}
        x & 0 & 0 & 0\\
        0 & y & 0 & 0\\
        0 & 0 & z & 0\\
        0 & 0 & 0 & 1\\
      \end{array}
    \right)
  \]
}

\card{
  How is it possible to combine two or more affine transformation matrices?
}{
  Multiply them together in the reverse order for which they are to be applied.
}

\card{
  How do you do transformation matrix powers?
}{
  You apply the matrix $n$ times, where $n$ is the power.
}

\card{
  What is the identity transformation?
}{
  A transformation that does nothing.
}

\end{document} 
