% Set the author and title of the compiled pdf
\hypersetup{
	pdftitle = {\Title},
	pdfauthor = {\Author}
}

\section{Terminology}

In order to talk about Strings in any meaningful way, we must first define
terminology that we can use to describe exactly what we mean. What follows is a
list of the terminology used throughout the course:

\begin{itemize}
	\item A {\bf symbol} is basically a letter. They are the basic component of
	all the data we use in the course. Examples include: {\it a, A, (, \$, 7}.
	\item An {\bf alphabet} is a collection of symbols that we can think of as a
	set. Example alphabets may include binary ({\it 0, 1}), latin letters ({\it
	a,\dots,z,A,\dots,Z}) etc.
	\item A {\bf String} is a collection of symbols grouped together, sometimes
	called a word. Examples include {\it ababa} and {\it 100101}.
	\item The {\it empty word} is a String consisting of no symbols. It is
	denoted by the letter $\epsilon$.
	\item {\bf Concentration} an operation that takes two Strings and combines
	them to create one longer String. For example concentrating {\it t} and {\it
	he} would create {\it the}. We can use the power notation to concentrate a
	String with itself any number of times. For example, ${ho}^3$ would give us
	$hohoho$.
	\item A {\bf language} is a collection of Strings that can be thought of as
	a set. Examples of languages couls be $\{\emptyset\}$, $\{\epsilon\}$,
	$\{hot,hotter,hottest\}$ or $\{a^n | n \in \mathbb{N}\}$.
\end{itemize}

We also have notation for describing generic instances of such entities:

\begin{center}
	\begin{tabular}{l l}
		{\bf Entity} & {\bf Generic notation}\\
		Symbol & $x, y, z$\\
		Alphabet & $\Sigma$\\
		String & $s, t, u$\\
		Language & $\mathcal{L}$\\
	\end{tabular}
\end{center}

\paragraph{Languages as sets} Since languages are thought of as sets, we can
perform all the usual set operations on them (see my {\tt COMP11120} notes for
more information on these operations). Languages can be concentrated as
described above, however, an interesting case is when a languages is subject to
concatenation zero times ($\mathcal{L}^0$), since this would return the empty
word $\epsilon$.

Sometimes, we may want to define any finite number of concatenations of a
language, and this is known as the {\bf Kleene star}. The notation is
$(\mathcal{L})^*$.

\marginpar{{\it Kleene} is pronounced like {\it genie}.}

\section{Describing languages}

\subsection{Describing languages through patterns}

A pattern describes a generic form that a set of Strings can take. If any String
from the set is compared with the pattern then it will match the pattern. Any
String from outside the set will not match the pattern.

For example, the pattern $(ab)^*$ would match Strings such as $\epsilon$, $ab$,
$abab$, $abab \dots ab$.

\marginpar{$\epsilon$ is matched here since it satisfies the pattern $(ab)^0$.
This comes about because, the Kleene star matches all concatenations of a
String.}

\subsection{Regular expressions}

The terms {\it pattern} and {\it regular expression} are pretty much synonomous.
The operators allowed in a regular expression are:

\begin{center}
	\begin{tabular}{>{\bfseries} l l}
		Empty pattern & The character $\emptyset$ is a pattern.\\
		Empty word & The character $\epsilon$ is a pattern.\\
		Letters & Every letter in $\Sigma$ is a pattern.\\
		\rowcolor{Gray}
		Concatenation & If $x$ and $y$ are patterns, then so is $(xy)$.\\
		\rowcolor{Gray}
		Alternative & If $x$ and $y$ are patterns, then so is $(x|y)$.\\
		\rowcolor{Gray}
		Kleene Star & If $x$ is a pattern, then so is $(x^*)$.
	\end{tabular}
\end{center}

\marginpar{If we were to analyse a pattern in a recursive fasion, then in order
to end our analysis, we would eventually have to find one of a selection of
{\it base cases}.The highlighted rows here represent {\it step cases} (that is
to say that at least another level of recursion is needed to finish our
analysis), while the un-highlighted lines are base cases.}

\subsubsection{Discarding brackets}

Just like in high-school mathematics, brackets can be discarded when unnecessary. For example, the pattern $((0|1)^*0)$ is equivalent to $(0|1)^*0$, and $(2|(1|0))$ is the same as $(2|1|0)$.

\subsubsection{Matching a regular expression}

As was implied above, we can define a regular expression by recursively
applying more operators to a `base case' until we have the desired pattern.

The following patterns match the following words:

\begin{center}
	\begin{tabular}{>{\bfseries} l l}
		Empty word & The empty word only matches the pattern $\epsilon$.\\
		% Carry on from page 17
	\end{tabular}
\end{center}