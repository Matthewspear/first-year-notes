\section*{Introduction}

The building of real-life computing systems, e.g. mobile phone, tv/video remote
control, internet shopping, air-traffic control, internet banking, etc., is
always a complex task. Mistakes can be very annoying, costly and sometimes life
threatening. Methods and techniques to support the building and understanding of
such systems are essential. This course unit provides an introduction to the
basic computer science ideas underlying such methods. It is also a part of, and
an introduction to, the Modelling and Rigorous Development theme.

\section*{Aims}

This course unit provides a first approach to answering the following questions.
What methods are there that can help understanding complicated systems or
programs? How can we make sure that a program does what we intend it to do? How
do computers go about recognizing pieces of text? If there are two ways of
solving the same problem, how can we compare them? How do we measure that one of
them gives the solution faster? How can we understand what computers can do in
principle, and are there problems that are not solvable by a computer?

\section*{Additional reading}

None.