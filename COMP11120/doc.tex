\documentclass{article}
% For the fancy header
\usepackage{fancyhdr}
% So we don't have to put \\ everywhere for new lines
\usepackage[parfill]{parskip}
% For sets e.g. \mathbb{Z}
\usepackage{amsfonts}
% For the 'split' envirnoment and symbols
\usepackage{amsmath,amssymb}
% For line brakes in tables
\usepackage{tabularx}
% For long division
\usepackage{polynom}
% For parse trees
\usepackage{proof}
% Gives us bigger margins on the right (and smaller on the left) for the margin
% paragraphs
\usepackage[left=2cm,
			top=3cm,
			right=5cm,
			bottom=3cm,
			marginparwidth=4cm,
			marginparsep=3mm]{geometry}

% Gives us a dot to use in parse trees. The phantom '|' symbols aren't shown but
% give us vertical (and a little bit of horizontal) space so the parse tree has
% the correct spacing.
\newcommand{\parsetreedot}{\phantom{|}\cdot\phantom{|}}

\pagestyle{fancyplain}

\author{Todd Davies, Chris Williamson}
\title{COMP11120 Notes}
\date{\today}

\begin{document}

\rhead{COMP11120 Notes}
\lhead{\today}

\maketitle

{\small Note, extra space has been allocated for the right hand margin to allow
for more extensive margin notes. Also, it gives you space to make your own
annotations and perhaps try some problems of your own.}

\tableofcontents
\newpage

\section{Discrete Structures}

\subsection{Terminology}

A {\it structure} consists of certain {\it sets}. It also contains {\it elements} of these sets, {\it operations} on these sets and {\it relations} on these sets.

\subsection{Number systems to learn}

The following number must be learnt:

\begin{tabularx}{\textwidth}{l X}
	$\mathbb{N}$ & The set of natural numbers (all whole numbers from $0$ to $\infty$)\\

	$\mathbb{Z}$ & The set of integers (all whole numbers from $-\infty$ to $\infty$)\\

	$\mathbb{Q}$ & The set of rational numbers (any integer divided by any other integer e.g. $\frac{5}{4}=1.25$)\\

	$\mathbb{R}$ & The set of real numbers (all finite and infinite decimal numbers)\\
\end{tabularx}

\subsubsection{Operations}

Each number system has a set of valid operations that can be performed on elements in that system. Number systems only contain operations that will produce an output that is still within the number system.

For example, the number system $\mathbb{N}$ contains the operations of addition and multiplication. This is because the summation of any two positive integers will {\it always} be a member of $\mathbb{N}$, and the same goes for multiplication.

However, you may be wondering why subtraction and division aren't included in this number system. This is because for some numbers, the result of subtraction or division won't be inside the set $\mathbb{N}$. An example of this would be subtracting $4$ from $2$. Even though both of the operands are inside $\mathbb{N}$, the answer isn't.

Different sets may have different operations available. For example, you can concentrate any two members of the set $\mathbb{S}tring$ and end up with another $\mathbb{S}tring$.

\paragraph{Types of operation}

Operations that have two operands either side of them are called {\it infix} operations. An example of an infix operation is addition. Infix operations are also referred to as {\it binary} operations since they have {\it two} operands.

\paragraph{Commutativity}

If an operation is commutative, the order of the operands doesn't matter. For example, addition is commutative since:

\[
	a + b = b + a
\]

Subtraction however, isn't commutative:

\[
	a - b \neq b - a
\]

\paragraph{Associativity}

An operation is associative if inserting or changing brackets doesn't change the outcome of the operation. For example, multiplication is associative since:

\[
	(a \times b) \times c = a \times (b \times c)
\]

An operation may only be commutative or associative if it is commutative or associative for {\it all} elements of the set that the operation supports.

\paragraph{Distinguished elements}

A set may contain distinguished elements that have strange effects on certain operations in the set. An example is the number 1. If we multiply {\it something} by 1, then the result will always be the same {\it something}. The same goes for 0 with addition. Because of this, we refer to 1 and 0 as distinguished elements of the set $\mathbb{Z}$.

\subsubsection{Relations}

Each of the sets $\mathbb{N, Q, Q, R}$ carries binary comparison relations $\leq$ and $<$. Different sets (such as $\mathbb{S}tring$) may have other relations, such as:

\begin{itemize}
	\item Is a section of
	\item Is an initial section of
	\item Occurs in
\end{itemize}

All relations return values in the set $\mathbb{B}ool$.

\subsection{Bases}

Conventionally, we count using base 10. Base 10 includes, you guessed it, ten different symbols from 0 through to 9.

Sometimes however, it is convenient to count using different bases. Popular bases include:

\begin{tabular}{l l l}
	{\bf Base $n$} & {\bf Member symbols} & {\bf Name}\\
	$n = 2$ & $\mathbb{Z}_2 = \{0, 1\}$ & Binary\\
	$n = 8$ & $\mathbb{Z}_8 = \{0, 1, 2, 3, 4, 5, 6, 7\}$ & Octal\\
	$n = 10$ & $\mathbb{Z}_{10} = \{0, 1, 2, 3, 4, 5, 6, 7\}$ & Decimal/Denary\\
	$n = 16$ & $\mathbb{Z}_{16} = \{0, 1, 2, \ldots ,9, A, B, C, D, E, F\}$ & Hexadecimal\\
\end{tabular}

\subsubsection{How to read numbers in any given base}

The formula for reading a number in a given base is as follows:

\[
	\sum_{i=0}^{k}a_{i}b^{i}
\]

Where the number you're trying to read takes the form $a_k, a_{k-1}, \ldots, a_2, a_1, a_0$ and $b$ is the base you're using.

\paragraph{Example 1}

Lets apply the formula to the base 10 number $27385$:

\[
	\begin{split}
		27385 &= (5 \times 10^0) + (8 \times 10^1) + (3 \times 10^2) + (7 \times 10^3) + (2 \times 10^4)\\
		      &= (5 \times 1) + (8 \times 10) + (3 \times 100) + (7 \times 1000) + (2 \times 10000)\\
		      &= 5 + 80 + 300 + 7000 + 20000\\
		      &= 27385
	\end{split}
\]

\paragraph{Example 2}

Lets apply the formula to the base 16 number $F00BA4$:

\[
	\begin{split}
		F00BA4 &= (4 \times 16^0) + (A \times 16^1) + (B \times 16^2) + (0 \times 16^3) + (0 \times 16^4) + (F \times 16^5)\\
		       &= (4 \times 16^0) + (10 \times 16) + (11 \times 256) + (0 \times 4096) + (0 \times 65536) + (15 \times 1048576)\\
		       &= 4 + 160 + 2816 + 0 + 0 + 15728640\\
		       &= 15731620
	\end{split}
\]

\subsubsection{Changing from base 10 to base $n$}

In order to change into base $n$ from base 10, we just repeatedly divide by $n$ and use the remainder as the value for base $n$. Here are a few examples:

\paragraph{Example 1}
Convert 893 into base 2.

\begin{center}
	\begin{tabular} {r l l l }
		$893 \div 2$ & = & 446 & r1\\
		$446 \div 2$ & = & 223 & r0\\
		$223 \div 2$ & = & 111 & r1\\
		$111 \div 2$ & = & 55 & r1\\
		$55 \div 2$ & = & 27 & r1\\
		$27 \div 2$ & = & 13 & r1\\
		$13 \div 2$ & = & 6 & r1\\
		$6 \div 2$ & = & 3 & r0\\
		$3 \div 2$ & = & 1 & r1\\
		$1 \div 2$ & = & 0 & r1\\
	\end{tabular}
\end{center}

Reading up from the bottom, we can see that the binary (base 2) representation is $1101111101$.

\paragraph{Example 2}
Convert 893 into base 9.

\begin{center}
	\begin{tabular} {r l l l }
		$893 \div 9$ & = & 99 & r2\\
		$99 \div 9$ & = & 11 & r0\\
		$11 \div 9$ & = & 1 & r2\\
		$1 \div 9$ & = & 0 & r1\\
	\end{tabular}
\end{center}

Reading up from the bottom, we can see that the nonal (base 9) representation is $1202$.

\paragraph{Example 2}
Convert 893 into base 16.

\begin{center}
	\begin{tabular} {r l l l }
		$893 \div 16$ & = & 55 & r13\\
		$55 \div 16$ & = & 3 & r7\\
		$3 \div 16$ & = & 0 & r3\\
	\end{tabular}
\end{center}

Reading up from the bottom, we can see that the hexadecimal (base 16) representation is $3,7,13$ or $37D$.

\subsection{A structure for the integers}
\subsubsection{The basic data}
The set $\mathbb{Z}$ of all integers: 

$\ldots,-3,-2,-1,0,1,2,3,\ldots$

includes the subset $\mathbb{N}$ of all natural numbers together with the negative integers. We will be using three basic binary operations on the carrier set $\mathbb{Z}$: addition, multiplication and subtraction.

The standard notation for there is +, x and - used as infix operations.

It is conventional to write {\it xy} as an abbreviation for {\it x}x{\it y}

Sometimes different notations are used, for example, in programming * is usually used.

These operations illustrate two properties which an arbitrary binary operation may or may not have.
 
Both + and x are {\bf commutative} but - is not.

\begin{enumerate}
\item For all integers {\it x,y} both, $x + y = y + x$ and $xy = yx$
\item There are integers {\it x,y} with $x - y \ne y - x$.
\end{enumerate}

Commutativity ensures that for most purposes the order in which the two arguments are consumed is irrelevant.

Both + and x are {\bf associative} but - is not.

\begin{enumerate}
\item For all integers {\it x,y,z} both $(x + y) + z = x + (y + z)$ , $(xy)z = x(yz)$.
\item There are integers {\it x,y,z} with $(x - y) - z \ne x - (y - z)$.
\end{enumerate}

Associativity ensures that for most purposes brackets are not needed to punctuate an expression. For instance, we can make sense of {\it x + y + z} and {\it xyz} since it doesn't matter where the brackets are, the resulting values are the same.

Below are the definitions of commutativity and associativity.

\marginpar{\raggedright N.b. Commutativity and associativity don't always have to go together. An example is the average of two numbers, what is that?}

\begin{enumerate}
  \item 
    $\circledast$ is {\bf commutative} if and only if
    \[
	a1 \circledast a2 = a2 \circledast a1
    \] 
    for all $a1,a2,\in,A$
  \item 
    $\circledast$ is {\bf associative} if and only if
    \[
	(a1 \circledast a2) \circledast a3 = a1 \circledast (a2 \circledast a3)
    \] 
    for all $a1,a2,a3,\in,A$
\end{enumerate}

Some numbers have special effects on operations in a set. Some examples are $0$ and $1$. When $0$ is added to a number, the result is still that number, and when multiplying a number by $1$, the result is still that number. This means for the set $\mathbb{Z}$, $0$ and $1$ are neutral elements for the operations addition and multiplication respectively.

\subsection{The formal language}

We use the formal language to talk about $\mathbb{Z}$. Expressions in the formal language are built up from atoms in a recursive fashion, so an expression may be $(x \circ y)$ where $x$ and $y$ are expressions and $\circ$ is one of the three operations ($+, -, \times$). The brackets are important in the formal language, since they ensure that strings can only be parsed in one way.

Strings that contain literals that aren't neutral elements or have invalid bracketing are not valid in the formal language. Examples may include:

\[
	(x + y + z)\\
\]
\[
	(x + 2)\\
\]
\[
	x((y
\]

We can use a parse tree to show how an expression is built up from it's identifiers ($a, b, c\ldots$), constants ($0, 1\ldots$) and operations ($+, -, \times\ldots$).

For example, the parse tree of $(x \times (y + z))$ is:

% See my SE question for more on these parse trees:
% http://tex.stackexchange.com/questions/141264/drawing-parse-trees

\[
	\infer[(\times)]{(x \times (y + z))}{
		x
		&
		\infer[(+)]{(y + z)}{
			y
			&
			z
		}
	}
\]

Often, especially with large parse trees, it's a pain to write so many
identifiers. Because of this, it is a convention to replace identifiers with
dots after they've been used once, like so:

\[
	\infer[(\times)]{\parsetreedot} {
		x
		&
		\infer[(+)]{\parsetreedot} {
			y
			&
			z
		}
	}
\]

In order to parse a parse tree, we must assign an appropriate value to each of
the 'leaves' of the tree (i.e. all the identifiers) and let the values trickle
down towards the root node of the tree where the evaluated answer will appear.

Lets use the above example again. Let $x = 4$, $y = 6$ and $z = 2$:

\[
	\infer[(\times)]{32} {
		4
		&
		\infer[(+)]{8} {
			6
			&
			2
		}
	}
\]

\marginpar{\raggedright The infix notation is when an operator is placed between
two operands, e.g. $2 + 2$. Other notations include the {\it prefix} and {\it postfix} notations.}

At this point, parse trees may seem very pointless, but this is because we're
doing very simple arithmetic. However, when we start to define other operators
that do unfamiliar things or don't use the {\it infix} notation,
then using parse trees can be a big help!

\subsection{The properties of sets}

All sets have some properties in common that we can use to manipulate them. Examples include membership, equality, inclusion etc.

\subsubsection{Set membership}

To indicate that an element is a member of a set, we use the $\in$ symbol. For example, to say that the element $true$ is a member of the set $\mathbb{B}$ool, we would write:

\[
	true \in \mathbb{B}ool
\]

Interestingly, we can parse this into English in many ways though. It could mean any of the following things:

\begin{itemize}
	\item $true$ is an element of the set $\mathbb{B}$ool
	\item $true$ is an member of the set $\mathbb{B}$ool
	\item $true$ is contained in $\mathbb{B}$ool
	\item $\mathbb{B}$ool contains $true$
\end{itemize}

Conversely, to indicate that an element is not a member of a set, we use the symbol $\notin$:

\[
	sheep \not\in \mathbb{B}ool
\]

Note that there is no concept of {\it order} or {\it repetition} in sets. This means that the following sets are all equal:

\[
	\{1,2,3\}
\]
\[
	\{2,3,1\}
\]
\[
	\{2,3,1,2\}
\]
\[
	\{3,3,3,3,3,2,1\}
\]

\subsubsection{Set equality}

Sets are equal if they have exactly the same members. Note that as far as sets are concerned, duplicate members are treated as just one member.

The notation for set equality is very easy. To say the set $X$ is equal to the set $Y$, we write:

\[
	X = Y
\]

However, you must understand that this is only true if for each element $a$ in
$X$ that element will also be contained in $Y$ and for each element $a$ in Y,
that element will also be contained within $X$:

\marginpar{\raggedright Sets with different descriptions can still be equal.
Convince yourself that the set of integers where $y^3 < y$ is equal to the set
of integers where $x < -1$}

\[
	\textrm{For each }a, a \in X \leftrightarrow a \in Y
\]

\subsubsection{Set inclusion}

If one set is a subset of another set, all the members of the first set are also found within the second set. In mathematical terms:

% TODO: Somebody else other than Todd veridy this is true and then delete this
% comment please. Peer review ftw!
\[
	\textrm{For each }a, a \in X \rightarrow a \in Y
\]

The notation for inclusion is $\subseteq$, so in the above example, we would
write:

\marginpar{\raggedright If $X \subseteq Y$ and $Y \subseteq X$ then what else an
we say about the relationship between $X$ and $Y$?}

\[
	X \subseteq Y
\]

\subsubsection{The empty set}

The empty set is a set that contains no members at all. It's symbol is
$\emptyset$.

Because the empty set has no members, it is a subset of all other sets:

\marginpar{\raggedright This is because otherwise $x in \emptyset$ would be true
for some $x$ where $x \not\in A$. This is impossible since there are no elements
in the $\emptyset$.}

\[
	\emptyset \in A
\]

\subsubsection{Singleton sets}

For any entity $a$, we can form a set consisting only of $a$:

\[
	\{a\}
\]

Be aware, a singleton set is not the same as the element contained within the set:

\[
	a \neq \{a\}
\]

\end{document}