\documentclass{article}
% For the fancy header
\usepackage{fancyhdr}
% So we don't have to put \\ everywhere for new lines
\usepackage[parfill]{parskip}
% For sets e.g. \mathbb{Z}
\usepackage{amsfonts}
% For line brakes in tables
\usepackage{tabularx}

\pagestyle{fancyplain}

\author{Todd Davies, Chris Williamson}
\title{COMP11120 Notes}
\date{\today}

\begin{document}

\rhead{COMP11120 Notes}
\lhead{\today}

\maketitle

\section{Discrete Structures}

\subsection{Terminology}

A {\it structure} consists of certain {\it sets}. It also contains {\it elements} of these sets, {\it operations} on these sets and {\it relations} on these sets.

\subsection{Number systems to learn}

The following number must be learnt:

\begin{tabularx}{\textwidth}{l X}
	$\mathbb{N}$ & The set of natural numbers (all whole numbers from $0$ to $\infty$)\\

	$\mathbb{Z}$ & The set of integers (all whole numbers from $-\infty$ to $\infty$)\\

	$\mathbb{Q}$ & The set of rational numbers (any integer divided by any other integer e.g. $\frac{5}{4}=1.25$)\\

	$\mathbb{R}$ & The set of real numbers (all finite and infinite decimal numbers)\\
\end{tabularx}

\subsubsection{Operations}

Each number system has a set of valid operations that can be performed on elements in that system. Number systems only contain operations that will produce an output that is still within the number system.

For example, the number system $\mathbb{N}$ contains the operations of addition and multiplication. This is because the summation of any two positive integers will {\it always} be a member of $\mathbb{N}$, and the same goes for multiplication.

However, you may be wondering why subtraction and division aren't included in this number system. This is because for some numbers, the result of subtraction or division won't be inside the set $\mathbb{N}$. An example of this would be subtracting $4$ from $2$. Even though both of the operands are inside $\mathbb{N}$, the answer isn't.

\end{document}
