\documentclass[frontgrid]{flacards}
\usepackage{tabularx}
\usepackage{color}

\geometry{landscape}

\definecolor{light-gray}{gray}{0.75}

\newcommand{\frontcard}[1]{\textcolor{light-gray}{\colorbox{light-gray}{$#1$}}}
\newcommand{\backcard}[1]{#1} 

\newcommand{\flashcard}[1]{% create new command for cards with blanks
    \card{% call the original \card command with twice the same argument (#1)
        \let\blank\frontcard% but let \blank behave like \frontcard the first time
        #1
    }{%
        \let\blank\backcard% and like \backcard the second time
        #1
    }%
}

\begin{document}

\pagesetup{3}{4} 

\card{
	Convert {\tt 0xF0} to decimal.
}{
	240
}

\card{
	Convert 18 to binary using 2's complement.
}{
	010010
}

\card{
	Where do the values that are fed into an operand come from when an
	instruction is executed using the three address style?
}{
	Memory.
	\vspace{1em}\\
	The resulting value is also copied to a destination address in memory.
}

\card{
	Why are registers faster than memory?
}{
	\begin{tabularx}{0.32\textwidth}{l X}
		- & Implemented using a flip flop or some other very fast volatile
		storage (rather than smaller, cheaper SRAM).\\
		- & Situated inside the processor, so there's less distance for the
		data to travel, which takes less time.\\
		- & Fewer of them so address decoding takes less time\\
		- & Data doesn't need to be transferred over a bus.\\
	\end{tabularx}
}

\card{
	How many registers does an ARM 32-bit processor have?
}{
	16
}

\card{
	What is the {\it one-address} style of instruction?
}{
	Where only one memory address may be used in any one instruction. The other
	operands must be registers.
}

\card{
	Describe the {\it load-store} style of instruction.
}{
	The only operations on memory are load and store operations. This means each
	instruction is very fast and very simple, but there are many instructions.
}

\card{
	How is code written in ARM Assembly run?
}{
	It is first assembled using an assembler into machine code. Then it is
	loaded into memory and executed sequentially.
}

\card{
	What is the instruction to load a value at a memory address into a register?
}{
	{\tt LDR <register>, <memory\_address\_alias>}
}

\card{
	What is the instruction to store a value in a register to a memory address?
}{
	{\tt STR <register>, <memory\_address\_alias>}
}

\card{
	What is the instruction to sum two numbers?
}{
	{\tt ADD <destination\_reg>, <operand\_1>, <operand\_2>}
}

\card{
	Can the {\tt MUL} command use literals? How else can you multiply literals?
}{
	No, it can't use literals, but you can load a literal into a register with
	{\tt MOV} and then multiply the register!
}

\card{
	What is the instruction to branch upon a condition?
}{
	{\tt B <condition>, <branch\_name>}
}

\card{
	What does the program counter do? What register is it?
}{
	It is used to store the memory address of the next instruction to be executed.
	\vspace{1em}\\
	It's register 15.
}

\card{
	When is the {\tt DEFW} command executed. What does it do? What's its syntax?
}{
	It is executed before the program runs.
	\vspace{1em}\\
	It stores a value at a memory address and assigns the address an alias.
	\vspace{1em}\\
	{\tt <alias> DEFW <value\_1>, ... <value\_n>}
}

\card{
	What does {\tt DEFB} do?
}{
    {\tt DEFB} stores a single byte in memory. If you give it a string, the
	whole string will be stored (in multiple bytes).
}

\card{
	What's the syntax of {\tt DEFS}?
}{
	{\tt <alias> DEFS <number\_of\_bytes>, <value\_of\_bytes>}
}

\card{
	What does {\tt STRB} do? What is it's syntax?
}{
	Stores the lowest eight bits of a register into memory.
	\vspace{1em}\\
	{\tt STRB <register>, <memory\_alias>}
}

\card{
	What does {\tt LDRB} do? What is it's syntax?
}{
	Loads the lowest eight bits of a specific memory address into a register. 
	The other bits in the register are set to zero.
	\vspace{1em}\\
	{\tt LDRB <register>, <memory\_alias>}
}

\flashcard{When using Little Endian, bytes are read from \blank{left} to \blank{right}.}

\flashcard{When using Big Endian, bytes are read from \blank{right} to \blank{left}.}

\card{How many bits are assigned to a literal in an ARM instruction}{12}

\card{
	What command should be used to load a literal into a {\it register}
}{
	{\tt LDR <register>, =<literal>}
	\vspace{1em}\\
	{\small Note, this is a pseudo instruction, that is converted to either
	{\tt MOV <register> \#<literal>} or it will define a constant and load that
	in from	memory.}
}

\card{
	How do you load an address into a register so you can use it as a pointer?
}{
	{\tt ADR <register>, <alias>}\\
	Now the {\tt <register>} will hold the memory location (and is therefore a
	pointer to) the alias.
}

\card{
	What does {\tt DEFW <number>} do?
}{
	It reserves a word of memory initialised to {\tt <number>}
}

\card{
	What does {\tt DEFB <value>} do?
}{
	It reserves a byte(s) of memory with the value {\tt <value>}. If a string is
	passed as the value, multiple bytes will be reserved, each with the value of
	a character.
}

\card{
	What does {\tt DEFS <size>, <fill>} do?
}{
    It reserves a block of memory {\tt <size>} bytes long initialised to the
	value {\tt <fill>}.
}

\card{
	What does {\tt ALIGN} do?
}{
	Leaves blank bytes in memory so that the next {\tt DEFINE} command will 
	start on a word boundary.
}

\card{
	What does {\tt ENTRY} do?
}{
	Sets the PC at the start of the program (i.e. dictates where the program
	should) start from.
}

\card{
	What does {\tt EQU} do?
}{
	{\tt EQU} allows us to name a literal. You can then refer to the literal 
	(still with a hash before it) by name in your code which is easy to read.
	\vspace{1em}\\
	{\tt <alias> EQU \#<value>}
}

\card{
	What are the four status flags provided by the ARM architecture?
}{
	\begin{tabularx}{0.5\textwidth}{l X}
		- & Negative\\
		- & Zero\\
		- & Carry\\
		- & Overflow\\
	\end{tabularx}
}

\card{
	How can you combine a CMP instruction with another instruction?
}{
	Append {\tt S} to an instruction. E.g.:\\
	{\tt SUBS R0, R1, R2}\\
	If the result in {\tt R0} was negative, then the negative flag would be set
	etc etc.
}

\card{
	What does {\tt RSB} do?
}{
	Reverse subtract. {\tt RSB R1, R0, \#0}:\\
	{\tt R1 = 0 - R0 = -R0}
}

\card{
	What does {\tt MLA} do?
}{
	Multiply and add. {\tt MLA R0, R1, R2, R3}:\\
	{\tt R0 = (R1 * R2) + R3}
}

% Look at my decoy \phantoms mwahaha!

\flashcard{
	\begin{tabularx}{0.32\textwidth}{l X}
		Condition code & Meaning (for cmp or subs)\\ \hline
		eq & \blank{Equal \phantom{1234567891011121314}}\\
		ne & \blank{Not equal \phantom{123456789101112}}\\
		ge & \blank{Signed greater than or equal}\\
		le & \blank{Signed less than or equal \phantom{12}}\\
		lt & \blank{Signed less than \phantom{123456}}\\
		gt & \blank{Signed greater than \phantom{1234}}\\
	\end{tabularx}
}

% -------------------
% Methods of addressing
% -------------------
\card{
	In the following instruction, what method of addressing is used, what will
	{\tt R0} contain and what will happen to {\tt R1}?
	\\\vspace{2em}
	{\tt LDR R0, [R1]}
}{
	This is called {\bf register-indirect addressing}.
	\\\vspace{2em}
        The value loaded into {\tt R0} will be the 32 bits stored at the memory
	address that is equal to the value in {\tt R1}.
	\\\vspace{2em}
	{\tt R1} won't be altered at all.
}

\card{
	In the following instruction, what method of addressing is used, what will
	{\tt R0} contain and what will happen to {\tt R1}?
	\\\vspace{2em}
	{\tt LDR	R0, [R1, \#4]}
}{
	This is called {\bf pre-indexed addressing}.
	\\\vspace{2em}
        The value loaded into {\tt R0} will be the 32 bits stored at the memory
	address that is equal to the value in {\tt R1 + 4}.
	\\\vspace{2em}
	{\tt R1} won't be altered at all.
}

\card{
	In the following instruction, what method of addressing is used, what will
	{\tt R0} contain and what will happen to {\tt R1}?
	\\\vspace{2em}
	{\tt LDR	R0, [R1, \#4]!}
}{
	This is called {\bf pre-indexed autoindexed addressing}.
	\\\vspace{2em}
        The value loaded into {\tt R0} will be the 32 bits stored at the memory
	address that is equal to the value in {\tt R1 + 4}.
	\\\vspace{2em}
	{\tt R1} will be incremented by {\tt 4} {\it before} the load operation.
}

\card{
	In the following instruction, what method of addressing is used, what will
	{\tt R0} contain and what will happen to {\tt R1}?
	\\\vspace{2em}
	{\tt LDR	R0, [R1], \#4}
}{
	This is called {\bf post-indexed autoindexed addressing}.
	\\\vspace{2em}
        The value loaded into {\tt R0} will be the 32 bits stored at the memory
	address that is equal to the value in {\tt R1 + 4}.
	\\\vspace{2em}
	{\tt R1} will be incremented by {\tt 4} {\it after} the load operation.
}

\card{
	In the following instruction, what method of addressing is used, what will
	{\tt R0} contain and what will happen to {\tt R1} and {\tt R2}?
	\\\vspace{2em}
	{\tt LDR	R0, [R1, R2]}
}{
	This is called {\bf register-indexed addressing}.
	\\\vspace{2em}
        The value loaded into {\tt R0} will be the 32 bits stored at the memory
	address that is equal to the value in {\tt R1 + R2}.
	\\\vspace{2em}
	{\tt R1} and {\tt R2} will stay the same.
}

\card{
	In the following instruction, what method of addressing is used, what will
	{\tt R0} contain and what will happen to {\tt R1} and {\tt R2}?
	\\\vspace{2em}
	{\tt LDR	R0, [R1, R2, LSL, \#2]}
}{
	This is called {\bf scaled register-indexed addressing}.
	\\\vspace{2em}
        The value loaded into {\tt R0} will be the 32 bits stored at the memory
	address that is equal to the value in {\tt R1 + (R2 * 4)}.
	\\\vspace{2em}
	{\tt R1} and {\tt R2} will stay the same.
}

% ---------------
% Strings
% ---------------

\card{
	How do you load a {\tt String} into a register?
}{
	\begin{tabularx}{0.5\textwidth}{X}
		{\tt msg DEFB "Hello"}\\
		{\tt ALIGN}\\
		{\tt ADRL R0, msg}\\
		{\tt SVC 3}
	\end{tabularx}
}

\card{
	What does the {\tt ADR} command do?
}{
	A psudo instruction that loads a program relative address into a register.
	Unlike {\tt ADRL} it only compiles to one instruction (so it's faster), but
	it has a smaller range of addressable addresses.
}

\card{
	What does the {\tt ADRL} instruction do?	
}{
	It is a psudo instruction that loads a program relative address into the
	register. It is compiled into two {\tt ADD} instructions.
}

\card{
	How do you find the length of a string defined by the alias {\tt message}?
}{
	\begin{tabularx}{0.5\textwidth}{X}
		{\tt 	ADRL R1, message}\\
		{\tt 	MOV R2, \#0}\\
		{\tt count LDRB R0, [R1, R2]}\\
		{\tt 	CMP R0, \#0}\\
		{\tt 	ADDNE R2, R2, \#1}\\
		{\tt 	BNE count}\\
		{\tt 	STR R2, length}
	\end{tabularx}
}

\card{
	What are the four ARM bit shifting/rotation instructions?
}{
	\begin{tabularx}{0.4\textwidth}{l X}
		{\bf Mnemonic} & {\bf Meaning} \\

		{\tt LSL} & Logical shift left \\

		{\tt LSR} & Logical shift right \\

		{\tt ASR} & Arithmetic shift right\\

		{\tt ROR} & Rotate Right
	\end{tabularx}
}

\card{
	What is the syntax of a shifting or rotation operation in ARM?
}{
	{\tt INSTRUCTION destination, operand, (\#)shift}
}

\card{
	How can {\tt LSL} be used to load words from a table/array in memory?
}{
	{\tt LDR R0, [R1, R2, LSL \#2]}
}

% --------
% Stacks
% --------

\card{
	What is the command to push something to the stack?
}{
	{\tt PUSH <register/literal>}
}

\card{
	What is the command to pop something off the stack?
}{
	{\tt POP <register/literal>}
}

\card{
	What is a different way of accessing the top value of the stack rather than
	using {\tt POP}?
}{
	{\tt LDR R1, [SP], \#4}
}

\card{
	What is a different way of adding a new element to the stack rather than
	using the {\tt PUSH} command?
}{
	{\tt STR R1, [SP, \#-4]!}
}

\card{
	How do you push or pop multiple elements from the stack without using the
	{\tt PUSH} or {\tt POP} commands?
}{
	{\tt STMFD SP!, {R0, R1}}\\
	{\tt LDMFD SP!, {R0, R1}}
}

%------------
% Methods
%------------

\card{
	What does the {\tt BL} command do?
}{
	\begin{tabularx}{0.32\textwidth}{X}
		- Moves the current value of the program counter (PC) into the link
		register (LR).\\
		- Branch to the label defined in the instruction by moving the memory
		address of that instruction into the PC.
	\end{tabularx}
}

\card{
	After a method called using the {\tt BL} command has finished executing, 
	what command does it use to tell the processor go back to what it was doing?
}{
	{\tt MOV PC, LR}
}

\card{
	If a method is using registers as temporary stores, what should it do before
	and after it executes? What instructions should it use to do this?
}{
	It should {\tt PUSH} the value of the registers to the stack and then
	{\tt POP} them again after:
	\begin{tabularx}{0.32\textwidth}{X}
		{\tt methodname}\\
		{\tt PUSH \{registers\_used\}}\\
		{\tt ; Do stuff}\\
		{\tt POP \{registers\_used\}}\\
		{\tt MOV PC, LR}
	\end{tabularx}
}

\card{
	How do you pass a parameter to a method?
}{
	Either put it in a register (this is the stupid way), or put it on the stack
	(this is a good way since you can pass lots of parameters like this.)
}

\card{
	How should a method access stacked parameters?
}{
	Just read the stack using the {\tt LDR} command and add an offset depending
	on what parameter you want. E.g:\\
	{\tt LDR R0, [SP, \#12]}
}

\card{
	What is a stack frame?
}{
	A stack frame is a set of values on the stack that relate to a single
	method. They may contain saved registers, temporary values used by the
	method, a pointer to the parent method etc.
}

%---------
% Switch statements
%---------

\card{
	How do you convert a Java switch statement into ARM Assembly?
}{
	\begin{tabularx}{0.32\textwidth}{X}
	1. Create a table with the values triggered by the case statement as the
	index of the table.\\
	2. Load the address of the table into a register ({\tt ADR R1, table})\\
	3. Get the value of the case variable in a register (we'll use {\tt R0}).\\
	3. Use the {\tt LDR} command to load the correct method to call
	({\tt LDR PC, [R1, R0, LSL, \#2]})\\
	4. Make sure you've got a default case\\
	5. Make sure you branch to the end of the case statement after each method.
	\end{tabularx}
}

\card{
	What are the ARM comparison operators for unsigned integers?
}{
	\begin{tabular}{l l}
		{\tt HI} & Higher \\
		{\tt HS} & Higher or same\\
		{\tt LO} & Lower\\
		{\tt LS} & Lower or same\\
	\end{tabular}
}

\card{
	How is a boolean represented in ARM assembly?
}{
	As an 8-bit set of values that is either {\tt 00000000} or {\tt 00000001}.
}

\card{
	How do you test if an ARM boolean is true or false?
}{
	{\tt CMP R1, \#0}\\
	If it's equal, then it's false.
}

\card{
	What are the logical operators that ARM has?
}{
	\begin{tabular}{l l}
		Logical AND & {\tt AND}\\
		Logical OR & {\tt ORR}\\
		Logical XOR & {\tt EOR}\\
		Bit clear & {\tt BIC}\\
		Logical NOT & {\tt MVN}
	\end{tabular}
}

\card{
	What is the syntax of an ARM logical operation such as {\tt EOR}?
}{
	{\tt EOR <reg\_1>, <reg\_2>}
}

\flashcard{
	Fill in the truth table:\\
	\begin{tabular}{|l l|c|}
		\hline
		{\tt A} & {\tt B} & {\tt BIC A, B}\\ \hline
		0 & 0 & \blank{0}\\
		0 & 1 & \blank{0}\\
		1 & 0 & \blank{1}\\
		1 & 1 & \blank{0}\\ \hline
	\end{tabular}
}

\card{
	What are the two methods of interfacing with peripherals?
}{
	\begin{tabular}{l}
	- Polling\\
	- Interrupts
	\end{tabular}
}

\card{
	When the CPU is polling a peripheral, what does the poll actually do?
}{
	\begin{tabularx}{0.32\textwidth}{X}
	- Read the status register\\
	- Use the {\tt TST} command to check the value of the status register
	against a bit pattern\\
	- If the test passes, then run a method to handle what to do next.
	\end{tabularx}
}

\card{
	Define 'memory mapped register'
}{
	A memory location that appears to be an actual memory location, but is
	actually situated inside a peripheral (so it is mapped to a different
	address).
}

\card{
	What does {\tt TST} do?
}{
	It performs a bitwise {\tt AND} on it's operands and then compares it to
	zero, setting the comparison flags as it does so.
}

\card{
	What is the syntax for the {\tt TST} instruction?
}{
	{\tt TST <register>, <pattern>}
}

\card{
	What happens when an interrupt occurs?
}{
	\begin{tabularx}{0.32\textwidth}{X}
		1. Stop the program execution (like a {\tt BL})\\
		2. Save important registers by stacking them (including the CSPR)\\
		3. Run the interrupt handler\\
		4. Restore the saved registers from the stack\\
		5. Copy the LR into the PC (and add 4) so the next instruction is
		executed.
	\end{tabularx}\\
	{\tiny Remember you can use {\tt STMFD} and {\tt LDMFD} to efficiently
	manage the stack}
}

\card{
	Write some sample code for checking a status register against a pattern.
}{
	\begin{tabular}{l}
		{\tt ADR R1, status\_reg}\\
		{\tt LDR R0, [R1]}\\
		{\tt TST R0, 0x40}\\
		{\tt BNE correct}\\
		{\tt B incorrect}
	\end{tabular}
}

\card{
	What is an interrupt vector table?
}{
	A table of pointers to various methods for handling different peripherals.
	It's implemented in the same way as a case statement is.
}

\card{
	What does {\tt SVC} stand for?
}{
	SuperVisor Call.
}

\card{
	What are the five SVC commands for? Where does it store or get it's data
	from?
}{
	\begin{tabular}{l l}
	0 & Output a character\\
	1 & Input a character\\
	2 & Stop execution\\
	3 & Output a string\\
	4 & Output an integer
	\end{tabular}\\
	{\tt R0}
}

\card{
	How do we get the parameter from {\tt SVC X}?
}{
	\begin{tabularx}{0.32\textwidth}{X}
		- Load the {\tt SVC X} instruction into a register with 
		{\tt LDR R0, [LR, \#-4]}\\
		- Clear the top 8 bits ({\tt BIC R0, R0, 0xFF000000})\\
		- Logic shift right by 8 places to get the parameter
		({\tt LSR R1, R0, \#8}).	
	\end{tabularx}
}

\card{
	What is direct memory access?
}{
	A hardware chip on the motherboard that allows peripherals to write directly
	to memory rather than going through the CPU first.
}

\card{
	List advantages and disadvantages of DMA.
}{
	\begin{tabularx}{0.32\textwidth}{r@{\hskip 0in}X r@{\hskip 0in}X}
	& {\bf Advantages} & & {\bf Disadvantages}\\
	\hline
	$\cdot$ & The load on the CPU is lessened. & $\cdot$ & The load on the
	memory and the peripheral is the same as before.\\
	& & $\cdot$ & The motherboard will be more complicated with the extra
	hardware.
	\end{tabularx}
}

%-----------
% Operating systems
%-----------

\card{
	What does the kernel do?
}{
	It abstracts the implementation of the hardware from the running programs.
	It also protects programs from each other by letting them only access memory
	locations that have been assigned to them.
}

\card{
	What is a kernel mode?
}{
	A kernel mode conveys privileges to a program. The default (user) mode only
	allows memory locations within the allocated space to be accessed. Other
	modes convey different privileges, e.g. the privileged mode which allows any
	address to be accessed.
}

%----------
% Methods of running code
%----------

\card{
	What are the five steps of compilation?
}{
	\begin{tabular}{l}
	1. Lexical analysis\\
	2. Syntaxic analysis\\
	3. Semantic analysis\\
	4. Code generation\\
	5. Optimisation
	\end{tabular}
}

\card{
	What are the pros and cons of an interpreted language?
}{
	\begin{tabularx}{0.32\textwidth}{r@{\hskip 0in}X r@{\hskip 0in}X}
	& {\bf Advantages} & & {\bf Disadvantages}\\
	\hline
	$\cdot$ & Simple to implement & $\cdot$ & Slow.\\
	$\cdot$ & Portable code & $\cdot$ & Hard to simulate a real computer on
	them.\\
	$\cdot$ & Increased security (since applications can be sandboxed). & &\\
	$\cdot$ & Ease of debugging since the state is within the interpreter & &\\
	\end{tabularx}
}

\card{
	Why does Java use zero address and one address instructions?
}{
	So that the bytecode is very small in size and can be quickly transferred
	over the Internet.
}

\flashcard{
	Code that is written using zero or one address instructions is called a
	\blank{stack machine}.
}

\card{
	What would a zero address {\tt ADD} instruction do?
}{
	\begin{tabular}{l}
	- Pop the two top values off the stack.\\
	- Add them together.\\
	- Push the result onto the stack again.
	\end{tabular}
}

\card{
	How would $x = (a + b) * c$ by represented in a stack machine?
}{
	\begin{tabular}{l}
	{\tt PUSH a}\\
	{\tt PUSH b}\\
	{\tt ADD}\\
	{\tt PUSH c}\\
	{\tt MUL}\\
	{\tt POP x}\\
	\end{tabular}
}

\card{
	What does Dynamic Class Loading do?
}{
	Only loads java classes just before they are executed to minimise memory
	useage and improve loading times of programs.
}

\card{
	What does the JIT compiler do?
}{
	It compiles classes into machine code in real time as they are loaded. It
	uses {\bf dynamic compilation} to refine it's compilation so that code that
	is run multiple times is optimised more than code that is run only once.
}

\card{
	What are the three areas of memory usage in Java programs?
}{
	The class area (containing method code and class variables), stack area
	(containing the stack, which includes method parameters, local variables and
	return pointers) and heap area (containing the objects used by the program).
}

\card{
	Where do Java objects reside in memory?
}{
	The heap.
}

\card{
	Where does the heap pointer point to?
}{
	The next free memory location in the heap.
}

\card{
	What are the three parts that each item in the heap is composed of?
}{
	A header (containing information about the object such as it's size),
	storage for instance variables and a table of method parameters that the
	object contains.
}

\card{
	What is a stale object?
}{
	An object in the heap that is no longer referenced by any pointer.
}

\card{
	What are the stages of garbage collection?
}{
	\begin{tabularx}{0.32\textwidth}{X}
	- Stop the program execution\\
	- Walk through the heap and mark objects as live or unreferenced.\\
	- Delete unreferenced objects in the heap.\\
	- Compact the heap by moving the live objects together (remembering to
	update the pointers to the moved objects).\\
	\end{tabularx}
}

\card{
	How do you do array access in ARM?
}{
	\begin{tabularx}{0.32\textwidth}{X}
	{\tt LDR R0, index}\\
	{\tt LDR R1, basePointer}\\
	{\tt LDR R2, [R1, R0, LSL \#2]}\\
	{\tt STR R2, element}
	\end{tabularx}
}

\end{document}