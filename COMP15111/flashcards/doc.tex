\documentclass[frontgrid]{flacards}
\usepackage{color}


\definecolor{light-gray}{gray}{0.75}

\newcommand{\frontcard}[1]{\textcolor{light-gray}{\colorbox{light-gray}{$#1$}}}
\newcommand{\backcard}[1]{#1} 

\newcommand{\flashcard}[1]{% create new command for cards with blanks
    \card{% call the original \card command with twice the same argument (#1)
        \let\blank\frontcard% but let \blank behave like \frontcard the first time
        #1
    }{%
        \let\blank\backcard% and like \backcard the second time
        #1
    }%
}

\begin{document}

\pagesetup{2}{4} 

% -------------------
% Methods of indexing
% -------------------
\card{
	In the following instruction, what method of indexing is used, what will
	{\tt R0} contain and what will happen to {\tt R1}?
	\\\vspace{2em}
	{\tt LDR	R0, [R1]}
}{
	This is called {\bf register-indirect addressing}.
	\\\vspace{2em}
        The value loaded into {\tt R0} will be the 32 bits stored at the memory
	address that is equal to the value in {\tt R1}.
	\\\vspace{2em}
	{\tt R1} won't be altered at all.
}

\card{
	In the following instruction, what method of indexing is used, what will
	{\tt R0} contain and what will happen to {\tt R1}?
	\\\vspace{2em}
	{\tt LDR	R0, [R1, \#4]}
}{
	This is called {\bf pre-indexed addressing}.
	\\\vspace{2em}
        The value loaded into {\tt R0} will be the 32 bits stored at the memory
	address that is equal to the value in {\tt R1 + 4}.
	\\\vspace{2em}
	{\tt R1} won't be altered at all.
}

\card{
	In the following instruction, what method of indexing is used, what will
	{\tt R0} contain and what will happen to {\tt R1}?
	\\\vspace{2em}
	{\tt LDR	R0, [R1, \#4]!}
}{
	This is called {\bf pre-indexed autoindexed addressing}.
	\\\vspace{2em}
        The value loaded into {\tt R0} will be the 32 bits stored at the memory
	address that is equal to the value in {\tt R1 + 4}.
	\\\vspace{2em}
	{\tt R1} will be incremented by {\tt 4} {\it before} the load operation.
}

\card{
	In the following instruction, what method of indexing is used, what will
	{\tt R0} contain and what will happen to {\tt R1}?
	\\\vspace{2em}
	{\tt LDR	R0, [R1], \#4}
}{
	This is called {\bf post-indexed autoindexed addressing}.
	\\\vspace{2em}
        The value loaded into {\tt R0} will be the 32 bits stored at the memory
	address that is equal to the value in {\tt R1 + 4}.
	\\\vspace{2em}
	{\tt R1} will be incremented by {\tt 4} {\it after} the load operation.
}

\card{
	In the following instruction, what method of indexing is used, what will
	{\tt R0} contain and what will happen to {\tt R1} and {\tt R2}?
	\\\vspace{2em}
	{\tt LDR	R0, [R1, R2]}
}{
	This is called {\bf register-indexed addressing}.
	\\\vspace{2em}
        The value loaded into {\tt R0} will be the 32 bits stored at the memory
	address that is equal to the value in {\tt R1 + R2}.
	\\\vspace{2em}
	{\tt R1} and {\tt R2} will stay the same.
}

\card{
	In the following instruction, what method of indexing is used, what will
	{\tt R0} contain and what will happen to {\tt R1} and {\tt R2}?
	\\\vspace{2em}
	{\tt LDR	R0, [R1, R2, LSL, \#2]}
}{
	This is called {\bf scaled register-indexed addressing}.
	\\\vspace{2em}
        The value loaded into {\tt R0} will be the 32 bits stored at the memory
	address that is equal to the value in {\tt R1 + (R2 * 4)}.
	\\\vspace{2em}
	{\tt R1} and {\tt R2} will stay the same.
}

\end{document} 
